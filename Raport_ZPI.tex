\documentclass[11pt]{aghdpl}
% \documentclass[en,11pt]{aghdpl}  % praca w języku angielskim

% Lista wszystkich języków stanowiących języki pozycji bibliograficznych użytych w pracy.
% (Zgodnie z zasadami tworzenia bibliografii każda pozycja powinna zostać utworzona zgodnie z zasadami języka, w którym dana publikacja została napisana.)
\usepackage[english,polish]{babel}

% Użyj polskiego łamania wyrazów (zamiast domyślnego angielskiego).
\usepackage{polski}

\usepackage[utf8]{inputenc}

% dodatkowe pakiety
%\usepackage[fleqn]{mathtools}

\usepackage{mathtools}
\usepackage{amsfonts}
\usepackage{amsmath}
\usepackage{amsthm}
\usepackage{graphics}
\usepackage{float}
\usepackage{epstopdf}
\usepackage{setspace}
\usepackage[export]{adjustbox}

\usepackage{array}
\usepackage{color}


\usepackage{csquotes}
% Ponieważ `csquotes` nie posiada polskiego stylu, można skorzystać z mocno zbliżonego stylu chorwackiego.
\DeclareQuoteAlias{croatian}{polish}

%\addbibresource{bibliografia.bib}

% Nie wyświetlaj wybranych pól.
%\AtEveryBibitem{\clearfield{note}}


% ------------------------
% --- < listingi > ---

% Użyj czcionki kroju Courier.
\usepackage{times}

\usepackage{listings}
\lstloadlanguages{TeX}

\lstset{
	literate={ą}{{\k{a}}}1
           {ć}{{\'c}}1
           {ę}{{\k{e}}}1
           {ó}{{\'o}}1
           {ń}{{\'n}}1
           {ł}{{\l{}}}1
           {ś}{{\'s}}1
           {ź}{{\'z}}1
           {ż}{{\.z}}1
           {Ą}{{\k{A}}}1
           {Ć}{{\'C}}1
           {Ę}{{\k{E}}}1
           {Ó}{{\'O}}1
           {Ń}{{\'N}}1
           {Ł}{{\L{}}}1
           {Ś}{{\'S}}1
           {Ź}{{\'Z}}1
           {Ż}{{\.Z}}1,
	basicstyle=\footnotesize\ttfamily,
}

\definecolor{bluekeywords}{rgb}{0,0,1}
\definecolor{greencomments}{rgb}{0,0.5,0}
\definecolor{redstrings}{rgb}{0.64,0.08,0.08}
\definecolor{xmlcomments}{rgb}{0.5,0.5,0.5}
\definecolor{types}{rgb}{0.17,0.57,0.68}

\usepackage{listings}
\lstset{language=[Sharp]C,
captionpos=b,
%numbers=left, %Nummerierung
%numberstyle=\tiny, % kleine Zeilennummern
frame=lines, % Oberhalb und unterhalb des Listings ist eine Linie
showspaces=false,
showtabs=false,
breaklines=true,
showstringspaces=false,
breakatwhitespace=true,
escapeinside={(*@}{@*)},
commentstyle=\color{greencomments},
morekeywords={partial, var, value, get, set},
keywordstyle=\color{bluekeywords},
stringstyle=\color{redstrings},
basicstyle=\ttfamily\small,
}

% ------------------------

\AtBeginDocument{
	\renewcommand{\tablename}{Tabela}
	\renewcommand{\figurename}{Rys.}
}

% ------------------------
% --- < tabele > ---

\usepackage{array}
\usepackage{tabularx}
\usepackage{multirow}
\usepackage{booktabs}
\usepackage{makecell}
\usepackage[flushleft]{threeparttable}

%\usepackage{subfig}
% defines the X column to use m (\parbox[c]) instead of p (`parbox[t]`)
\newcolumntype{C}[1]{>{\hsize=#1\hsize\centering\arraybackslash}X}


%---------------------------------------------------------------------------

\author{Krzysztof Belcarz}
\shortauthor{K.Belcarz}



\titlePL{Zarządzanie projektem informatycznym - raport końcowy}
\titleEN{}


\shorttitlePL{} % skrócona wersja tytułu jeśli jest bardzo długi
\shorttitleEN{}

\thesistype{Praca dyplomowa inżynierska}
%\thesistype{Master of Science Thesis}

\supervisor{}
%\supervisor{Marcin Szpyrka PhD, DSc}

\degreeprogramme{Informatyka Stosowana}

\date{2018}

\department{Katedra
Automatyki i Inżynierii Biomedycznej}
\faculty{Wydział Inżynierii Metali i Informatyki Przemysłowej}

\acknowledgements{}

\setlength{\cftsecnumwidth}{10mm}

%---------------------------------------------------------------------------
\setcounter{secnumdepth}{4}
\brokenpenalty=10000\relax

\begin{document}
\titlepages

% Ponowne zdefiniowanie stylu `plain`, aby usunąć numer strony z pierwszej strony spisu treści i poszczególnych rozdziałów.
\fancypagestyle{plain}
{
	% Usuń nagłówek i stopkę
	\fancyhf{}
	% Usuń linie.
	\renewcommand{\headrulewidth}{0pt}
	\renewcommand{\footrulewidth}{0pt}
}
\setcounter{tocdepth}{2}
\tableofcontents
\clearpage

\chapter{Wstęp}
\section{Karta projektu}
\subsection{Opis projektu (Project summary)}
Firma:
Projekt realizowany przez firmę ''SensorTech'' będącą liderem branży automotive z dziedziny sensoryki.
Kontekst:
Branża automotive posługuje się licznymi sensorami, które działają niezależnie od siebie, jendak jednym z kluczowych aspektów rozwoju pojazdów autonomicznych jest integracja tych sensorów.
\subsection{Cele (Objectives)}
Celem projektu jest przygotowanie systemu, który integruje pracę sensorów wizyjnych (w szczególności kamer) z platformą sprzętową "automotive grade" odpowiedzialną za realizację algorytmów i obliczeń, co pozwoli poprawić funkcjonowanie zaawansowanych systemów wspomagania kierowcy (ADAS). 
\subsection{Wstępny zakres projektu (Initial scope)}
System będzie miał za zadanie wysyłać do głównego modułu elektronicznego samochodu (ECU) informację dotyczącą zidentyfikowanego obiektu (zadanie klayfikacyjne sieci neuronowej, realizowane na platformie automotive grade) oraz odległości od niego (wyznaczanej za pośrednictwem kamer stereowizyjnych). Będzie on w pełni kompatybilny z każdym nowoczesnym samochodem dzięki dystrybucji danych poprzez powszechnie wykorzystywane magistrale. 
\subsection{Zespół (Team)}
W skład zespołu realizującego projekt wchodzą:
\begin{itemize}
\item[•] Project Manager,
\item[•] Buisness Analyst,
\item[•] Senior Software Engineer (neural networks specialist),
\item[•] Senior Software Engineer (datasets specialist),
\item[•] Senior Software Engineer (computer vision specialist),
\item[•] Software Engineer (neural networks specialist aid),
\item[•] Software Engineer (datasets specialist aid),
\item[•] Software Engineer (computer vision specialist aid),
\item[•] Debug Engineer
\item[•] Testing Engineer
\end{itemize}
\subsection{Założenia (Assumptions)}
Przyjmuje się założenia:
\begin{itemize}
\item[•] Współpraca z konsultantami z zespołu zajmującego się integracją hardware'ową i elektroniką
\item[•] Brak zmian w funkcjonalnościach systemu w czasie trwania projektu
\item[•] Testowanie systemu odbywa się w kontrolowanym środowisku, dobrze odzwierciedlającym ruch uliczny
\end{itemize}
\subsection{Harmonogram (Milestone schedule)}
Harmonogram projektu wygląda następująco:
\begin{itemize}
\item[•] Uruchomienie projektu
\item[•] Analiza wymagań
\item[•] Zaprojektowanie algorytmów
\item[•] Zakup niezbędnego sprzętu i jego konfiguracja
\item[•] Implementacja zaprojektowanych rozwiązań
\item[•] Testowanie
\item[•] Naprawa błędów
\item[•] Wdrożenie
\item[•] Przygotowanie dokumentacji
\end{itemize}
\subsection{Ryzyka (Risks)}
W trakcie realizacji projektu należy wziąć pod uwagę następujące czynniki ryzyka:
\begin{itemize}
\item[•] Brak możliwości uzyskania pożądanej jakości rezultatów, wynikający z niedeterministycznej natury działania elementów systemu ( sieci neuronowe )
\item[•] Wystąpienie problemów związanych z konfiguracją platformy sprzętowej
\item[•] Wypadki losowe związane z uszkodzeniem sprzętów (np. kamer) lub ich wady produkcyjnej  
\item[•] Nieefektywna komunikacja między zespołem realizującym projekt a zespołem, z którym konsultowane są jego elementy
\item[•] Niedostateczna wiedza członków zespołu
\end{itemize}
\subsection{Budżet (Budget)}


\chapter{Statystyka projektu}
\chapter{Zasoby i koszty}
\begin{figure}[H]
\centering
  \includegraphics[width=\linewidth]{zasoby}
  \caption{Zasoby i koszty}
  \label{zasoby}
\end{figure}

Charakterystyka zasobów:

\begin{center}
    \begin{tabular}{ | l | p{5cm} | p{5cm} |}
    \hline
    Nazwa zasobu & Wymagania & Odpowiedzialności \\ \hline
    Project Manager & 
    \begin{itemize}
	\item[•] stopień magistra informatyki, automatyki lub dziedziny pokrewnej 
	\item[•] znajomość języka angielskiego na poziomie zaawansowanym (minimum B2), potwierdzona stosownym dokumentem
	\item[•] umiejętność pracy w zespole
	\item[•] cechuje się  umiejętnością zarządzania,  zdecydowaniem, komunikatywnością, zdyscyplinowaniem, kreatywnością w rozwiązywaniu zadań 
	\item[•] umiejętność zarządzania projektem informatycznym potwierdzona ukończeniem stosownego kursu 
	\item[•] doświadczenie związane z zarządzaniem projektem informatycznym
	\end{itemize}	    	
     & 
	\begin{itemize}
	\item[•] zarządzanie zespołem
	\item[•] koordynacja działań osób biorących udział w projekcie
	\item[•] planowanie oraz wprowadzanie strategii działań w ramach projektu
	\item[•] sporządzanie stosownej dokumentacji związanej m.in. z analizą wymagań, analizą jakościową itp.
	\item[•] przygotowanie raportów okresowych i raportu końcowego
	\item[•] prowadzenie prezentacji i spotkań dotyczących projektu
	\end{itemize}	   
	\\ \hline
    Buisness Analyst & 
    \begin{itemize}
	\item[•] stopień magistra informatyki, automatyki lub dziedziny pokrewnej 
	\item[•] znajomość języka angielskiego na poziomie zaawansowanym (minimum B2), potwierdzona stosownym dokumentem
	\item[•] umiejętność pracy w zespole
	\item[•] cechuje się  umiejętnością zarządzania,  zdecydowaniem, komunikatywnością, zdyscyplinowaniem, kreatywnością w rozwiązywaniu zadań 
	\item[•]
	\end{itemize}	    	
     & 
	\begin{itemize}
	\item[•] analiza rynku pod kątem konkurencyjności rozwijanych rozwiązań
	\item[•] modelowanie i analiza procesów biznesowych
	\item[•] przygotowanie dokumentacji analityczno-projektowej
	\end{itemize}	   
	\\ \hline
    Senior Software Engineer & 
	\begin{itemize}
	\item[•] stopień magistra informatyki, automatyki lub dziedziny pokrewnej
	\item[•] znajomość języka angielskiego na poziomie zaawansowanym (minimum B2), potwierdzona stosownym dokumentem
	\item[•] umiejętność pracy w zespole
	\item[•] cechuje się komunikatywnością, zdyscyplinowaniem kreatywnością w rozwiązywaniu zadań 
	\item[•] fachowa wiedza z zakresu realizowanych działań
	\item[•] doświadczenie w pracy w dziedzinie związanej z realizowanym projektem 
	\end{itemize}	    
     & 
	\begin{itemize}
	\item[•] implementacja algorytmów 
	\item[•] komunikacja z osobami z innych działów biorącymi udział w projekcie
	\item[•] ścisła współpraca osobami odpowiedzialnymi za testowanie i debugowanie kodu
	\item[•] dbanie o zachowanie standardu i jakości kodu przez mniej doświadczonych współpracowników
	\end{itemize}	     
    \\ \hline   
    Software Engineer &
    \begin{itemize}
	\item[•] stopień magistra informatyki, automatyki lub dziedziny pokrewnej
	\item[•] znajomość języka angielskiego na poziomie zaawansowanym (minimum B2), potwierdzona stosownym dokumentem
	\item[•] umiejętność pracy w zespole
	\item[•] cechuje się komunikatywnością, zdyscyplinowaniem, kreatywnością w rozwiązywaniu zadań 
	\item[•] fachowa wiedza z zakresu realizowanych działań
	\end{itemize}	    
     & 
	\begin{itemize}
	\item[•] implementacja algorytmów 
	\item[•] dbanie o zachowanie standardu i jakości własnego kodu 
	\item[•] komunikacja z osobami z innych działów biorącymi udział w projekcie
	\end{itemize}
	\\ \hline
    Debug Engineer & 
    \begin{itemize}
	\item[•] stopień magistra informatyki, automatyki lub dziedziny pokrewnej
	\item[•] znajomość języka angielskiego na poziomie zaawansowanym (minimum B2), potwierdzona stosownym dokumentem
	\item[•] umiejętność pracy w zespole
	\item[•] cechuje się komunikatywnością, zdyscyplinowaniem, kreatywnością w rozwiązywaniu zadań 
	\item[•] znajomość frameworków do debugowania oprogramowania
	\end{itemize}	    
     & 
	\begin{itemize}
	\item[•]
	\item[•]
	\item[•]
	\end{itemize}
	\\ \hline
    Testing Engineer & 
    \begin{itemize}
	\item[•] stopień magistra informatyki, automatyki lub dziedziny pokrewnej
	\item[•] znajomość języka angielskiego na poziomie zaawansowanym (minimum B2), potwierdzona stosownym dokumentem
	\item[•] umiejętność pracy w zespole
	\item[•] cechuje się komunikatywnością, zdyscyplinowaniem, kreatywnością w rozwiązywaniu zadań 
	\item[•] znajomość frameworków do testowania oprogramowania
	\end{itemize}	    
     & 
	\begin{itemize}
	\item[•] zaprojektowanie i przeprowadzenie testów na wszystkich wymaganych poziomach
	\begin{itemize}
	\item[•] jednostkowe (inaczej modułowe, komponentów, unit-testy)
	\item[•] integracyjne
	\item[•] systemowe
	\item[•] akceptacyjne
	\end{itemize}
	\item[•] sporządzenie dokumentacji dotyczącej przeprowadzonych testów i ich wyników
	\end{itemize}
	\\ \hline
    \hline
    \end{tabular}
\end{center}

\chapter{Ścieżki komunikacji}
\section{Komunikacja w zespole}
Komunikacja w zespole realizowana będzie na kilku płaszczyznach. Podstawową metodą komunikacji jest bezpośrednia rozmowa, co jest możliwe w niewielkim zespole pracującym w tym samym biurze, ponadto narzędziami dostępnymi dla zespołu będą Skype for buisness oraz Outlook. Projekt realizowany będzie z zastosowaniem metodologii Scrum. Codziennie będą się odbywać spotaknia ''stand-up'', podczas których każdy ramowo streści jakimi zagadnieniami zajmował się wczoraj, jakie są tego efekty, jakie trudności napotkał oraz czym zajmować się będzie dzisiaj. Implementacja przebiegała będzie zgodnie ze standardowymi wytycznymi dla projektów działu R$\&$D (Research $\&$ Development), których dokumentacja jest powszechnie dostępna dla pracowników firmy poprzez intranet.
\section{Komunikacja z konsultantami z innych zespołów}
Komunikacja z członkami innych zespołów zaangażowanych w projekt odbywać się będzie w formie cyklicznych spotkań z wykorzystaniem Skype for buisness. Celem spotkań będzie weryfikacja działań, wokół których kompetencje zgromadzone są w innych zespołach.
\chapter{Etapy projektu}
\begin{figure}[H]
\centering
  \includegraphics[width=\linewidth]{etapy}
  \caption{Etapy projektu}
  \label{etapy}
\end{figure}

\chapter{Diagram sieciowy (następstwa zadań)}
\chapter{Struktura podziału pracy (Work Breakdown Structure) oraz przypisanie zasobów}
\begin{figure}[H]
\centering
  \includegraphics[width=\linewidth]{struktura1}
  \caption{Struktura podziału pracy 1}
  \label{struktura1}
\end{figure}
\begin{figure}[H]
\centering
  \includegraphics[width=\linewidth]{struktura2}
  \caption{Struktura podziału pracy 2}
  \label{struktura2}
\end{figure}
\chapter{Harmonogram realizacji projektu -  wykres Gantta}
\begin{figure}[H]
\centering
  \includegraphics[width=\linewidth]{gantt12}
  \caption{Wykres Gantta - etapy 1,2}
  \label{gantt12}
\end{figure}
\begin{figure}[H]
\centering
  \includegraphics[width=\linewidth]{gantt34}
  \caption{Wykres Gantta - etapy 3,4}
  \label{gantt34}
\end{figure}
\begin{figure}[H]
\centering
  \includegraphics[width=\linewidth]{gantt5678}
  \caption{Wykres Gantta - etapy 5,6,7,8}
  \label{gantt5678}
\end{figure}
\section{Ścieżka krytyczna}
\begin{figure}[H]
\centering
  \includegraphics[width=\linewidth]{sciezka_krytyczna1}
  \caption{Ścieżka krytyczna}
  \label{sciezka_krytyczna1}
\end{figure}
\begin{figure}[H]
\centering
  \includegraphics[width=\linewidth]{sciezka_krytyczna2}
  \caption{Ścieżka krytyczna}
  \label{sciezka_krytyczna2}
\end{figure}
\begin{figure}[H]
\centering
  \includegraphics[width=\linewidth]{sciezka_krytyczna3}
  \caption{Ścieżka krytyczna}
  \label{sciezka_krytyczna3}
\end{figure}
\chapter{Raporty}
\begin{figure}[H]
\centering
  \includegraphics[width=\linewidth]{raport_koszt}
  \caption{Raport - koszt}
  \label{raport_koszt}
\end{figure}
\begin{figure}[H]
\centering
  \includegraphics[width=\linewidth]{raport_praca}
  \caption{Raport - przegląd pracy}
  \label{raport_praca}
\end{figure}

\begin{figure}[H]
\centering
  \includegraphics[width=\linewidth]{raport_zadania_krytyczne}
  \caption{Raport - zadania krytyczne}
  \label{raport_zadania_krytyczne}
\end{figure}

\begin{figure}[H]
\centering
  \includegraphics[width=\linewidth]{raport_przeglad_projektu}
  \caption{Raport - przegląd projektu}
  \label{raport_przeglad_projektu}
\end{figure}
\chapter{Estymacja projektu informatycznego}
\section{Zarządzanie jakością}
\section{Ocena ryzyka}
\section{Ocena kosztów}
\include{Bibliografia}

\addcontentsline{toc}{chapter}{Bibliografia}
\listoffigures
\addcontentsline{toc}{chapter}{\listfigurename}
\listoftables
\addcontentsline{toc}{chapter}{\listtablename}


\end{document}
