\chapter{Wstęp}
\section{Karta projektu}
\subsection{Opis projektu (Project summary)}
Firma:
Projekt realizowany przez firmę ''SensorTech'' będącą liderem branży automotive z dziedziny sensoryki.
Kontekst:
Branża automotive posługuje się licznymi sensorami, które działają niezależnie od siebie, jendak jednym z kluczowych aspektów rozwoju pojazdów autonomicznych jest integracja tych sensorów.
\subsection{Cele (Objectives)}
Celem projektu jest przygotowanie systemu, który integruje pracę sensorów wizyjnych (w szczególności kamer) z platformą sprzętową "automotive grade" odpowiedzialną za realizację algorytmów i obliczeń, co pozwoli poprawić funkcjonowanie zaawansowanych systemów wspomagania kierowcy (ADAS). 
\subsection{Wstępny zakres projektu (Initial scope)}
System będzie miał za zadanie wysyłać do głównego modułu elektronicznego samochodu (ECU) informację dotyczącą zidentyfikowanego obiektu (zadanie klayfikacyjne sieci neuronowej, realizowane na platformie automotive grade) oraz odległości od niego (wyznaczanej za pośrednictwem kamer stereowizyjnych). Będzie on w pełni kompatybilny z każdym nowoczesnym samochodem dzięki dystrybucji danych poprzez powszechnie wykorzystywane magistrale. 
\subsection{Zespół (Team)}
W skład zespołu realizującego projekt wchodzą:
\begin{itemize}
\item[•] Project Manager,
\item[•] Buisness Analyst,
\item[•] Senior Software Engineer (neural networks specialist),
\item[•] Senior Software Engineer (datasets specialist),
\item[•] Senior Software Engineer (computer vision specialist),
\item[•] Software Engineer (neural networks specialist aid),
\item[•] Software Engineer (datasets specialist aid),
\item[•] Software Engineer (computer vision specialist aid),
\item[•] Debug Engineer
\item[•] Testing Engineer
\end{itemize}
\subsection{Założenia (Assumptions)}
Przyjmuje się założenia:
\begin{itemize}
\item[•] Współpraca z konsultantami z zespołu zajmującego się integracją hardware'ową i elektroniką
\item[•] Brak zmian w funkcjonalnościach systemu w czasie trwania projektu
\item[•] Testowanie systemu odbywa się w kontrolowanym środowisku, dobrze odzwierciedlającym ruch uliczny
\end{itemize}
\subsection{Harmonogram (Milestone schedule)}
Harmonogram projektu wygląda następująco:
\begin{itemize}
\item[•] Uruchomienie projektu
\item[•] Analiza wymagań
\item[•] Zaprojektowanie algorytmów
\item[•] Zakup niezbędnego sprzętu i jego konfiguracja
\item[•] Implementacja zaprojektowanych rozwiązań
\item[•] Testowanie
\item[•] Naprawa błędów
\item[•] Wdrożenie
\item[•] Przygotowanie dokumentacji
\end{itemize}
\subsection{Ryzyka (Risks)}
W trakcie realizacji projektu należy wziąć pod uwagę następujące czynniki ryzyka:
\begin{itemize}
\item[•] Brak możliwości uzyskania pożądanej jakości rezultatów, wynikający z niedeterministycznej natury działania elementów systemu ( sieci neuronowe )
\item[•] Wystąpienie problemów związanych z konfiguracją platformy sprzętowej
\item[•] Wypadki losowe związane z uszkodzeniem sprzętów (np. kamer) lub ich wady produkcyjnej  
\item[•] Nieefektywna komunikacja między zespołem realizującym projekt a zespołem, z którym konsultowane są jego elementy
\item[•] Niedostateczna wiedza członków zespołu
\end{itemize}
\subsection{Budżet (Budget)}


\chapter{Statystyka projektu}
\chapter{Zasoby i koszty}
Charakterystyka zasobów:

\begin{center}
    \begin{tabular}{ | l | l | p{5cm} |}
    \hline
    Nazwa zasobu & Wymagania & Odpowiedzialności \\ \hline
    Project Manager & - & - \\ \hline
    Senior Software Engineer & - & - \\ \hline
    Software Engineer & - & - \\ \hline
    Debug Engineer & - & - \\ \hline
    Testing Engineer & - & - \\ \hline
    \hline
    \end{tabular}
\end{center}

\chapter{Ścieżki komunikacji}
\section{Komunukacja w zespole}
Komunikacja w zespole realizowana będzie na kilku płaszczyznach. Podstawową metodą komunikacji jest bezpośrednia rozmowa, co jest możliwe w niewielkim zespole pracującym w tym samym biurze, ponadto narzędziami dostępnymi dla zespołu będą Skype for buisness oraz Outlook. Projekt realizowany będzie z zastosowaniem metodologii Scrum. Codziennie będą się odbywać spotaknia ''stand-up'', podczas których każdy ramowo streści jakimi zagadnieniami zajmował się wczoraj, jakie są tego efekty, jakie trudności napotkał oraz czym zajmować się będzie dzisiaj. Implementacja przebiegała będzie zgodnie ze standardowymi wytycznymi dla projektów działu R&D (Research & Development), których dokumentacja jest powszechnie dostępna dla pracowników firmy poprzez intranet.
\section{Komunikacja z konsultantami z innych zespołów}
Komunikacja z członkami innych zespołów zaangażowanych w projekt odbywać się będzie w formie cyklicznych spotkań z wykorzystaniem Skype for buisness. Celem spotkań będzie weryfikacja działań, wokół których kompetencje zgromadzone są w innych zespołach.
\chapter{Etapy projektu}
\chapter{Diagram sieciowy (następstwa zadań)}
\chapter{Struktura podziału pracy (Work Breakdown Structure) oraz przypisanie zasobów}
\chapter{Harmonogram realizacji projektu -  wykres Gantta}
\chapter{Raporty}
\chapter{Estymacja projektu informatycznego}
\section{Zarządzanie jakością}
\section{Ocena ryzyka}
\section{Ocena kosztów}








